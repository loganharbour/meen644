\documentclass{article}

\usepackage[margin=1in]{geometry}
\usepackage{graphicx} % Allow image/pdf includes
\usepackage{extramarks} % Extra header marks (continued on next page)
\usepackage{amsmath} % Math enhancements
\usepackage{amsthm} % Theorem typesetting
\usepackage{amssymb} % Extended symbol collection
\usepackage{tikz} % Graphical element creation
\usetikzlibrary{automata,positioning}
\usepackage{algpseudocode} % Algorithm layout
\usepackage{enumitem} % Enumerate (lists)
\usepackage{ragged2e} % Alternative alignment
\usepackage{gensymb} % Generic symbols (degree, etc)
\usepackage{empheq} % Allow \boxed around \begin{empheq}
\usepackage{color,soul} % Highlighting
\usepackage{booktabs} % Enhanced table creation
\usepackage{multirow} % Table multi row
\usepackage{mathtools} % Math enhancements
\usepackage{bm} % Bold math
\usepackage[mathscr]{euscript} % Script variables
\usepackage{cancel} % Cancel through text
\usepackage{color,soul} % Highlighting
\usepackage{mathtools}
\usepackage{multirow}
\usepackage{mathrsfs}
\usepackage{physics}
\usepackage{gensymb}
\usepackage{siunitx}
\usepackage{subcaption}
\usepackage[]{algorithm2e}
\usepackage{float}
\usepackage[cache=false]{minted}
\renewcommand{\MintedPygmentize}{/Users/logan/miniconda/bin/pygmentize}
\usepackage[scaled]{beramono}
\usepackage[T1]{fontenc}

\setlength\parindent{0pt} % No indents
\setlength{\parskip}{1em} % Paragraph skip

\newcommand{\vx}{\mathbf{x}} % x vector
\newcommand{\vy}{\mathbf{y}} % x vector

\newcommand{\pageTitle}{MEEN 644 - Homework 4}
\newcommand{\pageAuthor}{Logan Harbour}

\begin{document}

\title{\LARGE \textbf{\pageTitle} \vspace{-0.3cm}}
\author{\large \pageAuthor}
\date{\vspace{-0.6cm} \large \today \vspace{-0.4cm}}

\maketitle

\section*{Problem statement}

Consider a thin copper square plate of dimensions 0.5 m $\times$ 0.5 m. The temperature of the west and south edges are maintained at 50 $^\circ$C and the north edge is maintained at 100 $^\circ$C. The east edge is insulated. Using finite volume method, write a program to predict the steady-state temperature solution.

\begin{enumerate}[label=(\alph*)]
	\item \textbf{(35 points)} Set the over relaxation factor $\alpha$ from 1.00 to 1.40 in steps of 0.05 to identify $\alpha_\text{opt}$. Plot the number of iterations required for convergence for each $\alpha$.
	\item \textbf{(15 points)} Solve the same problem using $21^2, 25^2, 31^2$, and $41^2$ CVs, respectively. Plot the temperature at the center of the plate (0.25 m, 0.25 m) vs CVs.
	\item \textbf{(10 points)} Plot the steady state temperature contour in the 2D domain with the $41^2$ CV solution.
\end{enumerate}

\section*{Preliminaries}

\subsection*{Two-dimensional heat conduction}

With two-dimensional heat conduction with constant material properties, insulation on the right and prescribed temperatures on all other sides, we have the PDE
\begin{equation}
	\begin{cases}
		k \pdv{^2T}{x^2} + k \pdv{^2T}{y^2} = 0\,,\\
		T(x, 0) = T_B\,,\\
		T(0, y) = T_L\,,\\
		T(0, L_y) = T_T\,,\\
		-k \pdv{T}{x} \Big|_{x = L_x} = 0\,,
	\end{cases}
\end{equation}
where
\begin{align*}
	T_B & \equiv 50~^\circ\text{C}\,, & T_L & \equiv 50~^\circ\text{C}\,, & T_T & \equiv 100~^\circ\text{C}\,.\\
	k & \equiv 386~\text{W/m}~^\circ\text{C}\,, & L_x & \equiv 0.5~\text{m}\,, & L_y & \equiv 0.5~\text{m}\,.\\
\end{align*}

\subsection*{Control volume equations}

\begin{tikzpicture}[scale=2]
	\tikzset{dimen/.style={<->,>=latex,thin,every rectangle node/.style={fill=white,midway,font=\small}}}
	
	\draw (0,0) -- (2,0) -- (2,2) -- (0, 2) -- cycle;
	
	\filldraw (1, 1) circle (1pt);
	\filldraw (0, 1) circle (1pt);
	\filldraw (1, 0) circle (1pt);
	\filldraw (2, 1) circle (1pt);
	\filldraw (1, 2) circle (1pt);
	
	\node[below] at (1, 1) {$P_{i, j}$};
	%\foreach \i in {0, 1, 1.5}
	%\draw (\i, -0.1) -- (\i, 0.1);
	
	%\foreach \i in {0.5, 1.5}
	%\filldraw (\i, 0) circle (0.75pt);
	
	%\node[below] at (0.5, -0.05) {$\theta_{N-1}$};
	%\node[below] at (1.5, -0.05) {$\theta_N$};
	%\node[above] at (1.25, 0.4) {CV$_N$};
	%\node[above] at (0.5, 0.4) {CV$_{N-1}$};
	%\node[below] at (1.0, -0.1) {$w(N)$};
	
	%\draw [dimen] (0.5, 0.25) -- (1.5, 0.25) node {$\Delta x$};
	%\draw [dimen] (0.0, -0.5) -- (1.0, -0.5) node {$\Delta x$};
\end{tikzpicture}

\subsection*{Solving methodology}

\section*{Results}

\section*{Code listing}

For the implementation, we have the following files:
\begin{itemize}
	\item \texttt{Makefile} -- Allows for compiling the c++ project with \texttt{make}.
	\item \texttt{hwk4.cpp} -- Contains the \texttt{main()} function that is required by C that runs the cases requested in this problem set.
	\item \texttt{Flow2D.h} / \texttt{Flow2D.cpp} -- Contains the \texttt{Flow2D} class which is the solver for the 2D problem required in this homework.
	\item \texttt{Matrix.h} -- Contains the \texttt{Matrix} class which provides storage for a matrix with various standard matrix operations.
	\item \texttt{TriDiagonal.h} -- Contains the \texttt{TriDiagonal} class which provides storage for a tri-diagonal matrix including the TDMA solver found in the member function \texttt{solveTDMA()}.
	\item \texttt{plots.py} - Produces the plots in this report.
\end{itemize}

\subsection*{Makefile}
\inputminted[fontsize=\footnotesize]{Makefile}{../Makefile}

\subsection*{hwk4.cpp}
\inputminted[fontsize=\footnotesize]{c++}{../hwk4.cpp}

\subsection*{Flow2D.h}
\inputminted[fontsize=\footnotesize]{c++}{../Flow2D.h}

\subsection*{Flow2D.cpp}
\inputminted[fontsize=\footnotesize]{c++}{../Flow2D.cpp}

\subsection*{Matrix.h}
\inputminted[fontsize=\footnotesize]{c++}{../Matrix.h}

\subsection*{TriDiagonal.h}
\inputminted[fontsize=\footnotesize]{c++}{../TriDiagonal.h}

\subsection*{plots.py}
%\inputminted[fontsize=\footnotesize]{python}{../plots.py}

\end{document}
